
\documentclass[a4paper,12pt]{article} %% добавить leqno в [] для нумерации слева


\usepackage{ textcomp }
\usepackage{ gensymb }

%%%cross
\usepackage{ marvosym }

%%% Пустое множество
%%%\usepackage{ wasysym }

%%% Для команды mathds(Обозначение множества натур. чисел и т.д.)
\usepackage{ dsfont }


%%% Подчеркивание текста
\usepackage[normalem]{ulem}

%%% Работа с русским языком
\usepackage{cmap}					% поиск в PDF
\usepackage{mathtext} 				% русские буквы в формулах
\usepackage[T2A]{fontenc}			% кодировка
\usepackage[utf8]{inputenc}			% кодировка исходного текста
\usepackage[english,russian]{babel}	% локализация и переносы

%%% Дополнительная работа с математикой
\usepackage{amsmath,amsfonts,amssymb,amsthm,mathtools} %% AMS
\usepackage{icomma} % "Умная" запятая: $0,2$ --- число, $0, 2$ --- перечисление

%% Номера формул
%\mathtoolsset{showonlyrefs=true} % Показывать номера только у тех формул, на которые есть \eqref{} в тексте.

%% Шрифты
\usepackage{euscript}	 % Шрифт Евклид
\usepackage{mathrsfs} % Красивый матшрифт

%% Свои команды
%\DeclareMathOperator{\sgn}{\mathop{sgn}}

%% Перенос знаков в формулах (по Львовскому)
\newcommand*{\hm}[1]{#1\nobreak\discretionary{}
{\hbox{$\mathsurround=0pt #1$}}{}}

%%% Заголовок
\author{Богомолов Михаил. Весельев Александр}
\title{Telegram bot Aggregator }
\date{\today}

\begin{document}
	\maketitle
	\section*{Technical task}
		Aggregator bot представляет
		из себя бота-помощника в мессенджере Telegram. Язык программирования - 
		Python.
		Данный бот ставит перед собой задачу облегчения получения информации для
		пользователя из различных источников.
		Aggregator осуществляет выгрузку интересных для пользователя постов и
		сообщений из
		различных соц. сетей (на данный момент разрабатывается только поддержка 
		выгрузки постов и сообщений из vk.com)
	\section*{User-stories}
		\begin{enumerate}
			\item Как пользователь, я хочу сохранять посты, с той целью,
			что если они будут удалены из соц. сети я бы мог получить к ним доступ  
			\item Как пользователь, я хочу получать всю интересующую меня информацию
			из соц. сетей по запросу, чтобы она была собрана в одном месте, и мне не приходилось
			бы постоянно её искать в различных источниках
			\item Как обладатель нескольких страниц в разных соц. сетях, я хочу,
			чтобы вся информация собиралась в одном месте, и вместо просмотра 
			нескольких страниц мне достаточно было бы посмотреть её в одном Telegram
			чате.
		\end{enumerate}
\end{document}
